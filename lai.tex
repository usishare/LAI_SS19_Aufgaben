% layout and global options
\documentclass
[
  draft    = true,
  fontsize = 11pt,
  parskip  = half-,
  BCOR     = 0pt,
  DIV      = 11,
  ngerman
]
{scrartcl}

% default packages
\usepackage[utf8]{inputenc}
\usepackage[T1]{fontenc}
\usepackage{lmodern}
\usepackage{babel}
% extra packages
\usepackage{amsmath}
\usepackage{amssymb}
\usepackage{array}
\usepackage{enumerate}
\usepackage{graphicx}
\usepackage{ifthen}
\usepackage{siunitx}
\usepackage{tikz}
\usepackage{url}
\usepackage{xcolor}

% use comma as decimal separator
\sisetup{locale=DE, group-minimum-digits=4}

% -----
% defeq
% -----
%
% A vertically centered version of ':='
%
\newcommand{\defeq}{\mathrel{\mathop:}=}

% ---
% imp
% ---
%
% This macro typesets the imaginary part 'bi' of a complex number 'a+bi'.
%
% #1  b
%
\newcommand{\imp}[1]
{%
  % no b given
  \ifthenelse{\equal{#1}{}}
  {%
    \mathrm{i}%
  }%
  % b given
  {%
    #1\mskip0.5\thinmuskip\mathrm{i}%
  }%
}

% ------
% answer
% ------
%
% This environment typesets answers.
%
\newif{\ifanswers}
\newcommand{\hideanswers}{\answersfalse}
\newcommand{\showanswers}{\answerstrue}
%
\newenvironment{answer}
{%
  \begingroup
    \color{blue}%
    \sffamily
    \small
    \ifanswers
      \par
      \hrulefill
      \par
    \else
      \setbox0\vbox
      \bgroup
    \fi
}%
{%
    \ifanswers
      \relax
    \else
      \egroup
    \fi
  \endgroup
}%

% --------
% mytemize
% --------
%
% An itemize environment mith arrow symbols.
%
\newenvironment{mytemize}
{%
  \begingroup%
    % define arrow on first level
    \renewcommand{\labelitemi}
    {%
      \raisebox{0.6ex}
      {%
        \tikz\draw[line width=0.7pt, ->, >=latex, rounded corners=0.33ex]
                  (0ex, 1.9ex) -- (0ex, 1ex) -- (2.4ex, 1ex);%
      }%
    }%
    % use same arrow on each lavel
    \renewcommand {\labelitemii}  {\labelitemi}%
    \renewcommand {\labelitemiii} {\labelitemi}%
    \renewcommand {\labelitemiv}  {\labelitemi}%
    % set new margins
    \setlength {\labelsep}      {1.00ex}%
    \setlength {\labelwidth}    {2.75ex}%
    \setlength {\leftmargini}   {3.75ex}%
    \setlength {\leftmarginii}  {3.75ex}%
    \setlength {\leftmarginiii} {3.75ex}%
    \setlength {\leftmarginiv}  {3.75ex}%
    \vspace{-0.25\topsep}%
    \begin{itemize}%
      \setlength{\itemsep}{-0.5ex}%
}
{%
    \end{itemize}%
  \endgroup%
}

% ------------------------------------------------------------------------------
\begin{document}
% ------------------------------------------------------------------------------

\hideanswers
\showanswers

\allowdisplaybreaks

\thispagestyle{empty}
\begin{center}
  \vspace*{\fill}
  \normalsize
  \normalfont
  Aufgaben zur Veranstaltung\par
  \vspace{\baselineskip}
  \LARGE
  \bfseries
  Lineare Algebra für Informatiker\par
  \vspace{2\baselineskip}
  \normalsize
  \normalfont
  SS 2019\par
  \vspace*{\fill}
  \vspace*{\fill}
  \url{https://github.com/usishare/LAI_SS19_Aufgaben.git}
\end{center}

\clearpage
\pagenumbering{arabic}

% ---------------------
\paragraph{Aufgabe 1.1} \textit{(Mengen)}\par
% ---------------------
Sei $A=\{2,3\}$, $B=\{3,4\}$ und $C=\{2,3,4,5\}$. Bilden Sie folgende Mengen:
\begingroup
  \newcommand{\alg}{&\;}%
  \newcommand{\sep}{\;&\;}%
  \begin{align*}
        \text{a)}\sep A\cup B
    \alg\text{b)}\sep A\cap B
    \alg\text{c)}\sep (A\cup B)\cup C
    \alg\text{d)}\sep (A\cap B)\cap C
    \\
        \text{e)}\sep (A\cup B)\cap C
    \alg\text{f)}\sep (A\cap B)\cup C
    \alg\text{g)}\sep A\setminus B
    \alg\text{h)}\sep B\setminus A
    \\
        \text{i)}\sep (A\setminus B)\setminus C
    \alg\text{j)}\sep A\setminus(B\setminus C)
    \alg\text{k)}\sep A\times A
    \alg\text{l)}\sep A\times B
  \end{align*}
\endgroup

% ---------------------
\paragraph{Aufgabe 1.2} \textit{(Wahrheitstafeln)}
% ---------------------
\begin{enumerate}[a)]
  \item Stellen Sie jeweils die Wahrheitstafel auf:
        \begin{equation*}
          \text{a) } A\land(B\lor C)
          \qquad
          \text{b) } (A\land B)\lor(A\land C)
          \qquad
          \text{c) } A\lor(\lnot A)
        \end{equation*}
  \item Beweisen Sie -- falls möglich -- jeweils die Äquivalenz durch Vergleich
        der Wahrheitstafeln beider Seiten:
        \begin{alignat*}{2}
          &\text{a)}\quad & A\land(B\land C)&\Leftrightarrow(A\land B)\lor(A\land C)\\
          &\text{b)}\quad & A\lor(B\land(\lnot B))&\Leftrightarrow A
        \end{alignat*}
\end{enumerate}

% ---------------------
\paragraph{Aufgabe 1.3} \textit{(Reihen)}
% ---------------------
\begin{enumerate}[a)]
  \item Schreiben Sie die folgenden Reihen in der Form $\sum_{n=0}^{\infty}c_n$
        \begin{equation*}
          \begin{split}
            \text{a)}&\quad 3^4+5^5+7^6+9^7+\cdots \\[2ex]
            \text{b)}&\quad \frac{1}{2}+\frac{1}{3}+\frac{1}{4}+\frac{1}{5}+\frac{1}{6}+\cdots+\frac{1}{n+3} \\[2ex]
            \text{c)}&\quad \frac{1}{4}-\frac{1}{8}+\frac{1}{16}-\frac{1}{32}+\cdots
          \end{split}
        \end{equation*}
  \item Geben Sie jeweils den Zahlenwert der Summe an:
        \begin{alignat*}{2}
          \text{a)}&\quad\sum_{n=0}^{3}n^2
                   &\qquad\qquad
          \text{b)}&\quad\sum_{k=1}^{6}\frac{1}{k}-\frac{1}{k+1}
          \\[2ex]
          \text{c)}&\quad\sum_{n=1}^{2024}42
                   &\qquad\qquad
          \text{d)}&\quad\sum_{i=101}^{101}\ln(10^i)
        \end{alignat*}
\end{enumerate}

% ---------------------
\paragraph{Aufgabe 1.4} \textit{(Funktionen)}\par
% ---------------------
Gegeben seien folgende Funktionen:
\begin{equation*}
  \begin{split}
    f:\;&\mathbb{R}\to\mathbb{R}\;,\;x\mapsto3x-3\\
    g:\;&\mathbb{R}\to\mathbb{R}\;,\;x\mapsto(x+1)^2
  \end{split}
\end{equation*}
\begin{enumerate}[a)]
  \item Bestimmen Sie die Abbildungsvorschriften für $g\circ f$ und $f\circ g$.
        Geben Sie die Werte-, und Definitionsmengen von $f$, $g$, $f\circ g$ und $g\circ f$ an.
  \item Nachdem Sie die Definitions- und Zielmenge sinnvoll eingeschränkt haben,
        bestimmen Sie die Abbildungsvorschriften für $f^{-1}$ und $g^{-1}$.
\end{enumerate}

% ---------------------
\paragraph{Aufgabe 1.5} \textit{(Injektivität, Surjektivität und Bijektivität)}\par
% ---------------------
Untersuchen Sie $f$ jeweils auf Injektivität, Surjektivität und Bijektivität:
\begin{alignat*}{5}
  \text{a)}&\;\; & f:\;\mathbb{R}\to\mathbb{R}\;&,\;x\mapsto 3x-2
  &\quad&\qquad&
  \text{b)}&\;\; & f:\;\mathbb{Z}\to\mathbb{Z}\;&,\;x\mapsto 3x-2 
  \\[2ex]
  \text{c)}&\;\; & f:\;[0,\infty)\to\mathbb{R}\;&,\;x\mapsto x^2-2
  &\quad&\qquad&
  \text{d)}&\;\; & f:\;(1,\infty)\to[-1,\infty)\;&,\;x\mapsto x^2-2
  \\[1ex]
  \text{e)}&\;\; & f:\;(0,\infty)\to[2,\infty)\;&,\;x\mapsto\frac{x^2+1}{x}
  &\quad&\qquad&
  \text{f)}&\;\; & f:\;(\mathbb{R}\times\mathbb{R})\to\mathbb{R}\;&,\;(x,y)\mapsto x+y
\end{alignat*}

% ---------------------
\paragraph{Aufgabe 2.1} \textit{(Zahlenbereiche)}
% ---------------------
\begin{enumerate}[a)]
  \item Zeigen Sie, dass es keine rationale $x\in\mathbb{Q}$ gibt, sodass $x^2=2$ gilt.
  \item Für welche $n\in\mathbb{N}$ gilt $\sqrt{n}\in\mathbb{Q}$\,?
\end{enumerate}

% ---------------------
\paragraph{Aufgabe 2.2} \textit{(Körper)}\par
% ---------------------
Konstruieren Sie einen Körper aus den vier Elementen $\{0,1,a,b\}$.\par
{\itshape
Hinweis: Stellen Sie die Additions- und Multiplikationstafel auf,
und prüfen Sie die Körperaxiome.}

% ---------------------
\paragraph{Aufgabe 2.3} \textit{(Wiederholung Gruppen)}
% ---------------------
\begin{enumerate}[a)]
  \item Zeigen Sie, dass die Menge $G\defeq\mathbb{R}\setminus\{1\}$ mit der durch
        \begin{equation*}
          a\circ b\defeq a+b-ab\quad\forall a,b\in G
        \end{equation*}
        definierten Verknüpfung eine Gruppe ist.
  \item Lösen Sie in $G$ die Gleichung $a\circ x\circ6=-49$\,.\par
        {\itshape
        Hinweis: Falls $G$ eine Gruppe ist, müssen folgende Bedingungen gelten:}
        \begin{enumerate}[({G}1)]
          \item $\forall\;a,b\;:\;a,b\in G\;\Rightarrow\;a\circ b\in G$
          \item $\forall\;a,b,c\in G\;:\;(a\circ b)\circ c=a\circ(b\circ c)$
          \item $\exists\;e\in G\;:\;\forall\;a\in G\;:\;e\circ a=a$
          \item $\forall\;a\in G\;:\;\exists\;a^{-1}\in G\;:\;a^{-1}\circ a=e$
        \end{enumerate}
\end{enumerate}

% ---------------------
\paragraph{Aufgabe 2.4} \textit{(Komplexe Zahlen)}
% ---------------------
\begin{enumerate}[a)]
  \item Stellen Sie folgende komplexe Zahlen in der Form $x+\imp{y}$ mit
        $x,y\in\mathbb{R}$ dar:
        \begin{alignat*}{3}
          \text{a)}\;\;&(1-\imp{})^3+(1+\imp{})^3
          &\quad&\quad&
          \text{b)}\;\;&\frac{1}{1+\imp{3}}+\frac{1}{1-\imp{3}}
          \\[1ex]
          \text{c)}\;\;&\frac{1-\imp{}}{1+\imp{}}
          &\quad&\quad&
          \text{d)}\;\;&\left(\frac{1}{2}+\frac{\sqrt{3}}{2}\cdot\imp{}\right)^2
        \end{alignat*}
  \item Zeichnen Sie folgende komplexe Zahlen als Punkt der Ebene und
        berechnen Sie deren Beträge:
        \begin{equation*}
          \text{a)}\;\;z_1=7-\imp{9}
          \qquad
          \text{b)}\;\;z_2=1+\imp{}+\imp{}^2+\imp{}^3+\imp{}^4
          \qquad
          \text{c)}\;\;z_3=\frac{2+\sqrt{3}\cdot5}{2}
        \end{equation*}
\end{enumerate}

% ---------------------
\paragraph{Aufgabe 2.5} \textit{(Komplexe Zahlen)}\par
% ---------------------
Zeigen Sie, dass für alle komplexen Zahlen $x,y\in\mathbb{C}$ folgende
Zusammenhänge gelten:
\begin{equation*}
  \begin{split}
    |x+y|&\leq|x|+|y|\\
    |x\cdot y|&=|x|\cdot|y|
  \end{split}
\end{equation*}

% ---------------------
\paragraph{Aufgabe 3.1} \textit{(Ringe)}\par
% ---------------------
Stellen Sie die Additions- und Multiplikationstafeln der Ringe
$\mathbb{Z}_7$ und $\mathbb{Z}_4$ auf.
\begin{enumerate}[a)]
  \item Begründen Sie, ob es sich jeweils um einen Körper handelt oder nicht.
  \item Sind die Ringe jeweils Nullteilerfrei?
\end{enumerate}

% ---------------------
\paragraph{Aufgabe 3.2} \textit{(Polynomringe)}\par
% ---------------------
\begingroup
  \newcommand{\addop}{\oplus}%
  \newcommand{\mulop}{\otimes}%
  Sei $(R,\addop,\mulop)$ ein Ring und $f,g\in R[X]$ die folgenden zwei Polynome:
  \begin{equation*}
    \begin{split}
      f&=x^3+2x^2-4x-1\\
      g&=-2x^2+5
    \end{split}
  \end{equation*}
  Bestimmen Sie:
  \begin{equation*}
    \text{a)}\;\;f\addop g
    \qquad
    \text{b)}\;\;f\mulop g
    \qquad
    \text{c)}\;\;q,r\in R[X]\;\;\text{sodass}\;\;f=g\mulop q\addop r\text{ gilt}
  \end{equation*}
\endgroup

% ---------------------
\paragraph{Aufgabe 3.3} \textit{(Linearfaktoren)}\par
% ---------------------
Zerlegen Sie $f_1$ und $f_2$ in Linearfaktoren:
\begin{equation*}
  \begin{split}
    f_1&=x^2+\frac{100}{3}x+\frac{100}{3}\\[1ex]
    f_2&=x^3+\frac{3}{2}x^2-\frac{3}{2}x-1
  \end{split}
\end{equation*}
{\itshape
Hinweis: $-1$ ist eine Nullstelle von $f_1$
und $1$ bzw. $2$ sind Nullstellen von $f_2$.}

% ---------------------
\paragraph{Aufgabe 3.4} \textit{(Geradengleichung)}\par
% ---------------------
Es seien $t\in\mathbb{R}$ und $P\defeq(2\mid\frac{1}{4}\sqrt{2})$. Außerdem
\begin{equation*}
  \begin{split}
    G_1&\defeq\left\{(x,y)\in\mathbb{R}^2\mid2x+ty=4\right\}\\
    G_2&\defeq\left\{(x,y)\in\mathbb{R}^2\mid2x+4y=5\right\}
  \end{split}
\end{equation*}
\begin{enumerate}[a)]
  \item Für welche $t\in\mathbb{R}$ gilt $P\in G_1$\,?
  \item Bestimmen Sie in Abhängigkeit von $t$ den Schnittpunkt von $G_1$ und $G_2$.
        Schneiden sich $G_1$ und $G_2$ für jeden Wert von $t$?
\end{enumerate}

% ---------------------
\paragraph{Aufgabe 3.5} \textit{(Gruppen und Geraden)}\par
% ---------------------
Sei $G$ ist die Menge aller reeller Funktionen der Form $f(x)=ax+b$
mit $a\neq0$, also:
\begin{equation*}
  G\defeq
  \big\{
    f:\mathbb{R}\to\mathbb{R}
    \mid
    a,b\in\mathbb{R},a\neq0
    \;:\;
    f(x)=ax+b\quad\forall\;x\in\mathbb{R}
  \big\}
\end{equation*}
\begin{enumerate}[a)]
  \item Zeigen Sie, dass $G$ zusammen mit der Operation
        \begin{equation*}
          \circ:G\times G\to G
          \;,\;
          (f,g)\mapsto f\circ g
          \quad\text{mit}\quad
          (f\circ g)(x)=f(g(x))
        \end{equation*}
        eine Gruppe ist.
  \item Für welche $s,t\in\mathbb{R}$ gilt $sx+t=(ax+b)\circ(cx+d)$\,?
  \item Ist $G$ abelsch?
\end{enumerate}

% ---------------------
\paragraph{Aufgabe 4.1} \textit{(Gleichungssysteme)}\par
% ---------------------
Lösen Sie die folgenden Gleichungssysteme:
\begin{equation*}
  \begin{aligned}
    \text{a)}\quad x+y+z&=100\\
                   3x-2z&=4\\
                      5y&=4z
  \end{aligned}
  \qquad
  \qquad
  \begin{aligned}
    \text{b)}\quad x\sqrt{a}-y\sqrt{b}&=a+b\\
                                   x+y&=2\sqrt{a}\\
                                 \quad&\quad
  \end{aligned}
\end{equation*}

% ---------------------
\paragraph{Aufgabe 4.2} \textit{(Hähne, Hennen und Küken)}\par
% ---------------------
Wie viele Hähne, Hennen und Küken kann man für 100 Münzen kaufen, wenn
man insgesamt 100 Vögel haben will und ein Hahn 3 Münzen, eine Henne 5
Münzen und drei Küken eine Münze kosten? Die 100 Münzen sollen hierbei
vollständig verbraucht werden.
\begin{enumerate}[a)]
  \item Stellen Sie ein passendes lineares Gleichungssystem auf und
        geben Sie eine Lösung dieses Systems an, die das Problem löst.
  \item Ermitteln Sie die Menge aller Lösungen des Systems.
\end{enumerate}

% ---------------------
\paragraph{Aufgabe 4.3} \textit{(Gleichungssystem)}\par
% ---------------------
Gegeben sei das inhomogene lineare Gleichungssystem
\begin{equation*}
  \newcommand{\+}{&{}+{}&}%
  \renewcommand{\-}{&{}-{}&}%
  \renewcommand{\=}{&{}={}&}%
  \renewcommand{\.}{&{}~{}&}%
  \renewcommand{\|}{\text{\;\;}&}%
  \setlength{\arraycolsep}{0pt}%
  \begin{array}{|crcrcrcrcrcr}
  \|   a \+  3b \+ 2c \-   d \+ 4e \= 1 \\
  \|  4a \.     \+ 7c \+ 11d \+ 4e \= 2 \\
  \| -2a \+  8b \+ 3c \+  6d \+ 6e \= 3 \\
  \| 12a \- 18b \+ 2c \-  9d \- 6e \= 4
  \end{array}
\end{equation*}
\begin{enumerate}[a)]
  \item Bestimmen Sie den Lösungsraum des zugehörigen homogenen
        Gleichungssystems. Geben Sie insbesondere dessen Dimension und
        Rang an.
  \item Bestimmen Sie die Lösungsmenge dieses (unveränderten)
        Gleichungssystems.
\end{enumerate}

% ---------------------
\paragraph{Aufgabe 4.4} \textit{(Gleichung mit einer Variablen)}\par
% ---------------------
Bestimmen Sie in Abhängigkeit von $t\in\mathbb{R}$ die Lösungsmenge des
folgenden Gleichungssystems:
\begin{equation*}
  \newcommand{\+}{&{}+{}&}%
  \renewcommand{\-}{&{}-{}&}%
  \renewcommand{\=}{&{}={}&}%
  \renewcommand{\.}{&{}~{}&}%
  \renewcommand{\|}{\text{\;\;}&}%
  \setlength{\arraycolsep}{0pt}%
  \begin{array}{|crcrcrcrcrcr}
  \| ta \+  b \+  c \+  d \+  e \= 1 \\
  \|  a \+ tb \+  c \+  d \+  e \= 1 \\
  \|  a \+  b \+ tc \+  d \+  e \= 1 \\
  \|  a \+  b \+  c \+ td \+  e \= 1 \\
  \|  a \+  b \+  c \+  d \+ te \= 1
  \end{array}
\end{equation*}

% ---------------------
\paragraph{Aufgabe 4.5} \textit{(Dreh' am Rad)}\par
% ---------------------
Drei Zahnräder eines Getriebes haben zusammen 80 Zähne. Bei 10 Umdrehungen
des ersten Rades dreht sich das zweite Zahnrad 18 und das dritte 45 mal.\par
Wie viele Zähne hat jedes Rad?

% ---------------------
\paragraph{Aufgabe 5.1} \textit{(Matrizen)}\par
% ---------------------
Berechnen Sie jeweils die Matrix $C=A\cdot B$:
\begin{align*}
  \text{a)}\quad A_1&=
  \begin{pmatrix}
    2 &  5 & 3 \\
    1 & -2 & 0 \\
    0 &  1 & 4
  \end{pmatrix}
  &
  B_1&=
  \begin{pmatrix}
    1 & -3 \\
    2 &  4 \\
    0 &  2
  \end{pmatrix}
  \\[1ex]
  \text{b)}\quad A_2&=
  \begin{pmatrix}
    5 & 2 & 1 \\
    3 & 0 & 2
  \end{pmatrix}
  &
  B_2&=
  \begin{pmatrix}
    1 & 3 & 0 \\
    1 & 1 & 4 \\
    3 & 0 & 0
  \end{pmatrix}
  \\[1ex]
  \text{c)}\quad A_3&=
  \begin{pmatrix}
    2 & 1 & 2 & 1
  \end{pmatrix}
  &
  B_3&=
  \begin{pmatrix}
        4 \\
        3 \\
       -2 \\
        2
  \end{pmatrix}
  \\
  \text{d)}\quad A_4&=
  \begin{pmatrix}
        4 \\
        3 \\
       -2 \\
        2
  \end{pmatrix}
  &
  B_4&=
  \begin{pmatrix}
    2 & 1 & 2 & 1
  \end{pmatrix}
  \\
  \text{e)}\quad A_5&=
  \begin{pmatrix}
    2 & 7 & 6 \\
    4 & 8 & 2
  \end{pmatrix}^\text{t}
  &
  B_5&=
  \begin{pmatrix}
    1 & 0 \\
    0 & 1
  \end{pmatrix}
  \\[1ex]
  \text{f)}\quad A_6&=
  \begin{pmatrix}
    2
  \end{pmatrix}
  &
  B_6&=
  \begin{pmatrix}
    6
  \end{pmatrix}
  \\[1ex]
  \text{g)}\quad A_7&=
  \begin{pmatrix}
       \sqrt{5} &        0 &  -6 & \frac{3}{111} \\
             -2 &        3 &   8 &             2 \\
              0 & \sqrt{4} & 400 &            10 \\
    \frac{1}{2} &        1 &  -1 &             5
  \end{pmatrix}
  &
  B_7&=
  \begin{pmatrix}
    1 & 0 & 0 & 0 \\
    0 & 1 & 0 & 0 \\
    0 & 0 & 1 & 0 \\
    0 & 0 & 0 & 1
  \end{pmatrix}
\end{align*}

% ---------------------
\paragraph{Aufgabe 5.2} \textit{(Rechenregeln für Matrizen)}
% ---------------------
\begin{enumerate}[a)]
  \item Welche Eigenschaften müssen zwei beliebige Matrizen $A$ und $B$
        erfüllen um diese addieren zu können?
  \item Welche Eigenschaften müssen zwei beliebige Matrizen $C$ und $D$
        erfüllen um diese miteinander multiplizieren zu können?
  \item Welche Eigenschaften müssen erfüllt sein um eine Matrix $M$
        invertieren zu können?
\end{enumerate}

% ---------------------
\paragraph{Aufgabe 5.3} \textit{(Erweiterte Zeilenstufenform)}\par
% ---------------------
Formen Sie folgende Matrizen in die erweiterte Zeilenstufenform um:
\begin{equation*}
  \begin{split}
    R&=
    \begin{pmatrix}
       1 & 0 &  1 & -1 &  1 \\
       1 & 1 &  1 &  1 &  0 \\
      -1 & 1 & -1 &  1 & -1 \\
       0 & 1 & -1 &  0 &  1
    \end{pmatrix}
    \in M_{4\times5}(\mathbb{R})
    \\[1ex]
    S&=
    \begin{pmatrix}
      1+\imp{} &       0 &  5 \\
             5 &  \imp{} & 10 \\
            -1 & \imp{2} & -3
    \end{pmatrix}
    \in M_{3\times3}(\mathbb{C})
    \\[1ex]
    T&=
    \begin{pmatrix}
       1 & 6 & 3 \\
      14 & 3 & 1 \\
       2 & 5 & 1
    \end{pmatrix}
    \in M_{3\times3}(\mathbb{Z}_5)
  \end{split}
\end{equation*}

% ---------------------
\paragraph{Aufgabe 5.4} \textit{(Ring)}\par
% ---------------------
Sei $R=\big(M_{2\times2}(\mathbb{R}),\oplus,\otimes\big)$,
wobei die Verknüpfungen $\oplus$ und $\otimes$ jeweils für die kanonische
Addition bzw. Multiplikation von Matrizen stehen.\par
Zeigen Sie, dass $R$ ein Ring ist. Dabei kann davon ausgegangen werden,
dass die Distributivität und Assoziativität bei beiden Verknüpfungen
erfüllt ist.

% ---------------------
\paragraph{Aufgabe 6.1} \textit{(Linearkombinationen)}\par
% ---------------------
Stellen Sie den Vektor $w=(1,0,0)^\text{t}\in\mathbb{R}^3$ jeweils als
Linearkombination der drei Vektoren $v_1$, $v_2$ und $v_3$ dar.
\begin{alignat*}{3}
    \text{a)}\quad
    v_1&=
    \begin{pmatrix}
      1 \\
      0 \\
      1
    \end{pmatrix}
    \qquad&
    v_2&=
    \begin{pmatrix}
      7 \\
      3 \\
      1
    \end{pmatrix}
    \qquad&
    v_3&=
    \begin{pmatrix}
       4 \\
       3 \\
      -1
    \end{pmatrix}
    \\[1ex]
    \text{b)}\quad
    v_1&=
    \begin{pmatrix}
      2 \\
      1 \\
      0
    \end{pmatrix}
    \qquad&
    v_2&=
    \begin{pmatrix}
      3 \\
      0 \\
      5
    \end{pmatrix}
    \qquad&
    v_3&=
    \begin{pmatrix}
      -1 \\
       4 \\
      -1
    \end{pmatrix}
\end{alignat*}

% ---------------------
\paragraph{Aufgabe 6.2} \textit{(Lineare Un- bzw. Abhängigkeit)}\par
% ---------------------
Gegeben seien folgende vier Vektoren des $\mathbb{R}^3$:
\begin{equation*}
  v_1=
  \begin{pmatrix}
    1 \\
    1 \\
    0
  \end{pmatrix}
  \qquad
  v_2=
  \begin{pmatrix}
    3 \\
    0 \\
    2
  \end{pmatrix}
  \qquad
  v_3=
  \begin{pmatrix}
    -3 \\
    -3 \\
    -4
  \end{pmatrix}
  \qquad
  v_4=
  \begin{pmatrix}
    1 \\
    1 \\
    1
  \end{pmatrix}
\end{equation*}
\begin{enumerate}[a)]
  \item Sind $v_1$, $v_2$, $v_3$ und $v_4$ linear unabhängig?
  \item Welche dreielementigen Teilmengen von $\{v_1,v_2,v_3,v_4\}$ sind linear unabhängig?
  \item Stellen Sie $w=(3,6,2)^\text{t}$ als Linearkombination von $v_1$, $v_2$ und $v_4$ dar.
\end{enumerate}

% ---------------------
\paragraph{Aufgabe 6.3} \textit{(Vektorraum)}\par
% ---------------------
Sei $V$ ein Vektorraum über einem Körper $K$.
Zeigen Sie, dass wenn für $\lambda\in K$ und $v\in V$ die Gleichung
$\lambda v=0$ gilt, dann auch $\lambda=0$ oder $v=0$ gelten muss.

% ---------------------
\paragraph{Aufgabe 6.4} \textit{(Gleichungssysteme)}\par
% ---------------------
Bestimmen Sie die Lösungsmengen der folgenden Gleichungssysteme:
\begingroup
  \newcommand{\exnum}[1]{\text{\makebox[2em][r]{#1)}\quad}}%
  \newcommand{\LGSA}
  {%
    \begingroup
      \setlength{\arraycolsep}{1pt}
      \begin{array}{l|rcrcrcrl}
         \exnum{a}&\; -19x & + & 19y & + &  z & = & 0 & \\
                  &\;    x &   &     & + & 7z & = & 0 & \\
                  &\;  -2x & + &  7y & + & 3z & = & 0 &   
      \end{array}
    \endgroup
  }%
  \newcommand{\LGSB}
  {%
    \begingroup
      \setlength{\arraycolsep}{1pt}
      \begin{array}{l|rcrcrl}
         \exnum{b}&\;       2x & + &             (4+\imp{})y & = & 10       & \\
                  &\; \imp{3}x & + & (-\frac{3}{2}+\imp{6})y & = & \imp{15} &
      \end{array}
    \endgroup
  }%
  \newcommand{\LGSC}
  {%
    \begingroup
      \setlength{\arraycolsep}{1pt}
      \begin{array}{l|rcrcrcrcrl}
         \exnum{c}&\; 2x_{1} & + &  x_{2} & - & 2x_{3} & + & 3x_{4} & = & 1 & \\
                  &\; 3x_{1} & + & 2x_{2} & - &  x_{3} & + & 2x_{4} & = & 4 & \\
                  &\; 3x_{1} & + & 3x_{2} & + & 3x_{3} & - & 3x_{4} & = & 5 &   
      \end{array}
    \endgroup
  }%
  \begin{equation*}
    \begin{array}{ll}
      \LGSA & \LGSC \\
            &       \\
      \LGSB &
    \end{array}
  \end{equation*}
\endgroup

% ---------------------
\paragraph{Aufgabe 6.5} \textit{(Lösungsmengen)}\par
% ---------------------
Entscheiden und begründen Sie ob folgende Aussagen wahr oder falsch sind:
\begin{enumerate}[a)]
  \item Jedes lineare Gleichungssystem hat eine Lösungsmenge.
  \item Jedes homogene Gleichungssystem mit mehr Unbekannten als Gleichungen
        hat mindestens 2 Lösungen.
  \item Jedes inhomogene lineare Gleichungssystem mit mehr Gleichungen
        als Unbekannten ist unlösbar.
  \item Wenn das inhomogene Gleichungssystem mehrere Lösungsmöglichkeiten
        hat, hat auch das dazugehörige homogene Gleichungssystem mehrere
        Lösungen.
\end{enumerate}

% ---------------------
\paragraph{Aufgabe 6.6} \textit{(Modelleisenbahn)}\par
% ---------------------
Auf einer geschlossenen Bahn von \SI{440}{\centi\metre} Länge treffen
sich zwei Körper mit einer gleichgerichteter Bewegung alle \SI{20}{\minute},
bei entegegengesetzter Bewegung alle \SI{5}{\minute}. Wie groß sind die
Geschwindigkeiten beider Körper?

% ---------------------
\paragraph{Aufgabe 7.1} \textit{(Erzeugendensystem)}\par
% ---------------------
Mit $e_i$ seien im Folgenden die Einheisvektoren des $\mathbb{R}^n$ bezeichnet, also:
\begin{equation*}
  \begin{split}
    e_1&=(1,0,0,\ldots,0,0)^\text{t}\\
    e_2&=(0,1,0,\ldots,0,0)^\text{t}\\
 \vdots&                            \\
    e_n&=(0,0,0,\ldots,0,1)^\text{t}
  \end{split}
\end{equation*}
Welche der folgenden Teilmengen des $\mathbb{R}^n$ sind Erzeugendensysteme?
\begin{enumerate}[a)]
  \item $\big\{e_1+e_2,\;e_2+e_3,\;\ldots,\;e_{n-1}+e_n\big\}$
  \item $\big\{e_n,\;e_{n-1},\;e_{n-2},\;\ldots,\;e_2,\;e_1,\;e_1+e_2\big\}$
  \item $\big\{e_1,\;e_2,\;e_1+e_3+e_3+e_4,\;0\big\}$ für $n=5$
  \item $\left\{e_1+e_2+e_3,\;
               e_2+e_3+e_4,\;
               e_3+e_4+e_5,\;
                   e_4+e_5,\;
                       e_5
               \right\}$ für $n=5$
\end{enumerate}

% ---------------------
\paragraph{Aufgabe 7.2} \textit{(Basis)}
% ---------------------
\begin{enumerate}[a)]
  \item Für welche $t\in\mathbb{R}$ bilden die Vektoren $u$, $v$ und $w$ eine
        Basis des $\mathbb{R}^3$\,?
        \begin{equation*}
          u=
          \begin{pmatrix}
              1 \\
              2 \\
            t+2
          \end{pmatrix}
          \qquad
          v=
          \begin{pmatrix}
             -1 \\
            t+1 \\
              t
          \end{pmatrix}
          \qquad
          w=
          \begin{pmatrix}
            0 \\
            t \\
            1
          \end{pmatrix}
        \end{equation*}
  \item Es sei $\{v_1,v_2\}$ eine Basis eines zweidimensionalen Vektorraums $V$.
        Untersuchen Sie, für welche Zahlen $s,t\in\mathbb{R}$ auch die beiden Vektoren
        $w_1=sv_1+v_2$ und $w_2=v_1+tv_2$ eine Basis von $V$ bilden.
\end{enumerate}

% ---------------------
\paragraph{Aufgabe 7.3} \textit{(Basis)}\par
% ---------------------
Konstruiere für die folgenden Vektorräume jeweils eine Basis:
\begin{equation*}
  \begin{split}
    U_1&=\left\{(x,y,z)\in\mathbb{R}^3\mid x+2y+z=0\right\}\\
    U_2&=\left\{(x_1,x_2,x_3,x_4)\in\mathbb{R}^4\mid x_1+x_2+x_3=0\;\land\; x_1+x_3+x_4=0\right\}
  \end{split}
\end{equation*}

% ---------------------
\paragraph{Aufgabe 7.4} \textit{(Austauschsatz)}\par
% ---------------------
Im Folgenden sei $B$ eine Basis eines Vektorraums $V$ und $C$ eine Menge
linear unabhängiger Vektoren in $V$.
\begin{equation*}
  B=
  \left\{
    \begin{pmatrix}
      1 \\
      2 \\
      0 \\
      1
    \end{pmatrix}
    ,
    \begin{pmatrix}
      1 \\
      1 \\
      0 \\
      1
    \end{pmatrix}
    ,
    \begin{pmatrix}
      -2 \\
      10 \\
       0 \\
       3
    \end{pmatrix}
  \right\}
  \qquad
  C=
  \left\{
    \begin{pmatrix}
      -2 \\
      25 \\
       0 \\
       8
    \end{pmatrix}
    ,
    \begin{pmatrix}
      -8 \\
      28 \\
       0 \\
       7
    \end{pmatrix}
  \right\}
\end{equation*}
\begin{enumerate}[a)]
  \item Ersetzen Sie einen Vektor aus $B$ durch einen Vektor aus $C$.
  \item Ergänzen Sie die neue Basis $B$ zu einer Basis des $\mathbb{R}^4$.
\end{enumerate}

% ---------------------
\paragraph{Aufgabe 7.5} \textit{(Begriffe)}\par
% ---------------------
Definieren Sie folgende Begriffe in eigenen Worten:
\begin{itemize}
  \item linear abhängig
  \item linear unabhngig
  \item Erzeugendensystem
  \item Basis
\end{itemize}

% ---------------------
\paragraph{Aufgabe 8.1} \textit{(Untervektorräume)}\par
% ---------------------
Untersuchen Sie, welche der folgenden Mengen Untervektorräume des
$\mathbb{R}^2$ sind:
\begin{alignat*}{3}
  U_1&=\left\{(x,y)\in\mathbb{R}^2\mid y=2x\right\}
  \qquad&\qquad
  U_2&=\left\{(x,y)\in\mathbb{R}^2\mid x^4+y^4=0\right\}
  \\
  U_3&=\left\{(x,y)\in\mathbb{R}^2\mid y=2+x\right\}
  \qquad&\qquad
  U_4&=\left\{(x,y)\in\mathbb{R}^2\mid x\geq y\right\}
\end{alignat*}

% ---------------------
\paragraph{Aufgabe 8.2} \textit{(Untervektorräume)}\par
% ---------------------
Untersuchen Sie, für welche $c\in\mathbb{R}$ die Menge
$U_c$ ein Untervektorraum des $\mathbb{R}^4$ ist:
\begin{equation*}
  U_c=\left\{(x_1,x_2,x_3,x_4)\in\mathbb{R}^4\mid x_1+x_2+x_3+x_4=c\right\}
\end{equation*}

% ---------------------
\paragraph{Aufgabe 8.3} \textit{(Rang)}\par
% ---------------------
Gegeben sei folgende lineare Abbildung:
\begin{equation*}
  \varphi:\mathbb{R}^3\to\mathbb{R}^4\;,\;x\mapsto Ax
  \quad
  \text{mit}
  \quad
  A=
  \begin{pmatrix}
    1 & 0 & 2 \\
    1 & 1 & 1 \\
    0 & 1 & 1 \\
    1 & 0 & 2
  \end{pmatrix}
\end{equation*}
\begin{enumerate}[a)]
  \item Berechnen Sie den Rang von $A$.
  \item Ist $\varphi$ injektiv oder surjektiv?
  \item Zeigen Sie, dass der Vektor $b=(2,1,3,2)^\text{t}$ im Bild von
        $\varphi$ liegt.
\end{enumerate}

% ---------------------
\paragraph{Aufgabe 8.4} \textit{(Komplement)}\par
% ---------------------
Gegeben seien folgende Vektoren des $\mathbb{R}^3$:
\begin{equation*}
  v_1=
  \begin{pmatrix}
     1 \\
    -2 \\
     0
  \end{pmatrix}
  \qquad
  v_2=
  \begin{pmatrix}
    0 \\
    0 \\
    2
  \end{pmatrix}
  \qquad
  v_3=
  \begin{pmatrix}
    -2 \\
     4 \\
     2
  \end{pmatrix}
\end{equation*}
\begin{enumerate}[a)]
  \item Geben Sie eine Basis für den von $v_1$, $v_2$ und $v_3$
        erzeugten Untervektorraum $U$ von $\mathbb{R}^3$ an.
  \item Bestimmen Sie eine Basis des Komplements $U_c$ von $U$.
\end{enumerate}

% ---------------------
\paragraph{Aufgabe 8.5} \textit{(Lineare Abbildungen)}\par
% ---------------------
Welche der folgenden Abbildungen sind $\mathbb{R}$-linear?
\begin{alignat*}{4}
  f_1:&\;\; & \mathbb{R}^2&\to\mathbb{R}   & \quad&,\quad & (x,y)&\mapsto3x+y             \\
  f_2:&\;\; & \mathbb{R}^2&\to\mathbb{R}^3 & \quad&,\quad & (x,y)&\mapsto(x^2,\;y^2,\;xy) \\
  f_3:&\;\; &   \mathbb{R}&\to\mathbb{R}^2 & \quad&,\quad &     x&\mapsto(0,0)
\end{alignat*}

% ---------------------
\paragraph{Aufgabe 8.6} \textit{(Kern und Bild)}\par
% ---------------------
Es sei folgende $\mathbb{R}$-lineare Abbildung $f$ definiert durch:
\begin{equation*}
  f:\;\mathbb{R}^3\to\mathbb{R}^3\;,\quad
  \begin{pmatrix}
    x \\
    y \\
    z
  \end{pmatrix}
  \mapsto
  \begin{pmatrix}
    -1 & 0 &  2 \\
     1 & 6 &  4 \\
     3 & 3 & -3
  \end{pmatrix}
  \cdot
  \begin{pmatrix}
    x \\
    y \\
    z
  \end{pmatrix}
\end{equation*}
Konstruieren Sie jeweils eine Basis von $\ker(f)$ und
$\operatorname{bild}(f)$.

% ---------------------
\paragraph{Aufgabe 9.1} \textit{(Dimensionssatz)}\par
% ---------------------
Gegeben seien folgende Unterräume des $\mathbb{R}^3$:
\begin{equation*}
  S=
  \left\{
    \begin{pmatrix}
      1 \\
      3 \\
      4
    \end{pmatrix}
    ,
    \begin{pmatrix}
                0 \\
      \frac{1}{2} \\
               10
    \end{pmatrix}
    ,
    \begin{pmatrix}
      -10 \\
      -28 \\
        0
    \end{pmatrix}
  \right\}
  \qquad
  T=
  \left\{
    \begin{pmatrix}
      -2 \\
       0 \\
       3
    \end{pmatrix}
    ,
    \begin{pmatrix}
      7 \\
      1 \\
      0
    \end{pmatrix}
    ,
    \begin{pmatrix}
      1 \\
      1 \\
      9
    \end{pmatrix}
  \right\}
\end{equation*}
Berechnen Sie $\dim(S)$, $\dim(T)$, $\dim(S+T)$ und $\dim(S\cap T)$.

% ---------------------
\paragraph{Aufgabe 9.2} \textit{(Selbstinverse Matrizen)}\par
% ---------------------
Bestimmen Sie alle Matritzen der Form
\begin{equation*}
  F=
  \begin{pmatrix}
    a &  b \\
    c & -a
  \end{pmatrix}
  \in M_{2\times2}(\mathbb{R})
  \quad\text{mit}\quad
  a\neq0\;,
\end{equation*}
die zu sich selbstinvers sind.\par
{\itshape
Hinweis: selbstinvers bedeutet, dass $A\cdot A=I$ gilt.}

% ---------------------
\paragraph{Aufgabe 9.3} \textit{(Isomorphismus)}\par
% ---------------------
Beschreiben Sie was ein Isomorphismus ist, und geben Sie ein Beispiel an.

% ---------------------
\paragraph{Aufgabe 9.4} \textit{(Darstellungsmatrizen)}\par
% ---------------------
Gegeben sei folgende $\mathbb{R}$-lineare Abbildung:
\begin{equation*}
  f:\;\mathbb{R}^3\to\mathbb{R}^3\;,\;
  (
    x_1
    ,
    x_2
    ,
    x_3
  )^\text{t}
  \mapsto
  (
    x_1-x_2+x_3
    \;,\;
    8x_1-4x_2
    \;,\;
    -2x_1+2x_2-2x_3
  )^\text{t}
\end{equation*}
Berechnen Sie die Matrix $M(f,B,C)$ für
\begin{enumerate}[a)]
  \item $B=C=\left\{(1,0,0)^\text{t},(0,1,0)^\text{t},(0,0,1)^\text{t}\right\}$
  \item $B=C=\left\{(-1,0,1)^\text{t},(-1,2,1)^\text{t},(2,0,-1)^\text{t}\right\}$
\end{enumerate}

% ---------------------
\paragraph{Aufgabe 9.5} \textit{(Basistransformation)}\par
% ---------------------
Gegeben seien folgende Basen des $\mathbb{R}^3$ bzw. des $\mathbb{R}^2$:
\begin{alignat*}{3}
  B&=
  \left\{
    \begin{pmatrix}
       17 \\
      -25 \\
        1
    \end{pmatrix}
    ,
    \begin{pmatrix}
      0 \\
      1 \\
      0
    \end{pmatrix}
    ,
    \begin{pmatrix}
      16 \\
       0 \\
       1
    \end{pmatrix}
  \right\}
  \qquad&\qquad
  B'&=
  \left\{
    \begin{pmatrix}
      1 \\
      0 \\
      0
    \end{pmatrix}
    ,
    \begin{pmatrix}
      0 \\
      1 \\
      0
    \end{pmatrix}
    ,
    \begin{pmatrix}
      16 \\
       2 \\
       1
    \end{pmatrix}
  \right\}
  \\[1ex]
  C&=
  \left\{
    \begin{pmatrix}
      1 \\
      0
    \end{pmatrix}
    ,
    \begin{pmatrix}
      0 \\
      1
    \end{pmatrix}
  \right\}
  \qquad&\qquad
  C'&=
  \left\{
    \begin{pmatrix}
      3 \\
      7
    \end{pmatrix}
    ,
    \begin{pmatrix}
      2 \\
      5
    \end{pmatrix}
  \right\}
\end{alignat*}
Ferner sei $f:\mathbb{R}^3\to\mathbb{R}^2$ eine $\mathbb{R}$-lineare
Abbildung mit der Matrix
\begin{equation*}
  M_B^C(f)=
  \begin{pmatrix}
    3 & 2 & -1 \\
    7 & 5 &  6
  \end{pmatrix}.
\end{equation*}
Berechnen Sie die Matrizen
\begin{equation*}
  S=M(\operatorname{id},C',C)
  \qquad
  T=M(\operatorname{id},B',B)
  \qquad
  U=M(f,B',C')
\end{equation*}

% ----------------------
\paragraph{Aufgabe 10.1} \textit{(Permutationen)}\par
% ----------------------
Gegeben seien folgende Permutationen aus der Gruppe $S_{10}$:
\begin{equation*}
  \begin{split}
    \sigma_1&=
    \begin{pmatrix}
      1 & 2 & 3 & 4 & 5 & 6 &  7 & 8 & 9 & 10 \\
      3 & 4 & 5 & 6 & 1 & 2 & 10 & 9 & 8 &  7
    \end{pmatrix}
    \\
    \sigma_2&=
    \langle1,2,3,4,5\rangle
    \langle3,4,5,6,7\rangle
    \langle5,6,7,8,9,10\rangle
  \end{split}
\end{equation*}
Bestimmen Sie für beide Permutationen
\begin{enumerate}[a)]
  \item die kanonische Zyklendarstellung
  \item eine Darstellung durch Transpositionen
  \item das Signum
\end{enumerate}

% ----------------------
\paragraph{Aufgabe 10.2} \textit{(Determinanten berechnen)}\par
% ----------------------
Gegeben seien folgende Matrizen:
\begin{equation*}
  R=
  \begin{pmatrix}
     1 & 2 & 3 \\
     0 & 1 & 2 \\
    -4 & 0 & 1
  \end{pmatrix}
  \in M_{3\times3}(\mathbb{R})
  \qquad
  S=
  \begin{pmatrix}
     5 & 2 & 0 \\
    -1 & 3 & 1 \\
     2 & 0 & 1
  \end{pmatrix}
  \in M_{3\times3}(\mathbb{R})
\end{equation*}
Berechnen Sie $\det(R)$, $\det(-R)$, $\det(-S)$, $\det(S\cdot R)$,
$\det(R\cdot S)$ und $\det(S^\text{t})$.

% ----------------------
\paragraph{Aufgabe 10.3} \textit{(Invertierbar?)}\par
% ----------------------
Welche der folgenden Matrizen sind invertierbar:
\begin{alignat*}{3}
  A&=
  \begin{pmatrix}
    10 &          51 \\
     2 &         -38 \\
     0 & \frac{1}{6}
  \end{pmatrix}
  \qquad&\qquad
  B&=
  \begin{pmatrix}
     0 &  1 & -10 &  0 \\
    -1 & -3 &   1 &  0 \\
     1 &  0 &   1 &  5 \\
    -2 &  3 &   0 & -1
  \end{pmatrix}
  \\[1ex]
  C&=
  \begin{pmatrix}
     2 &  -2 & 2 & -2 &   2 \\
    51 & -17 & 0 & -1 & 100 \\
     0 &   2 & 0 &  2 &   0 \\
     1 &   2 & 3 &  4 &   5 \\
     1 &  -2 & 1 & -2 &   1
  \end{pmatrix}
  \qquad&\qquad
  D&=
  \begin{pmatrix}
    0 &   0 &  2 &  1 &           -1 \\
    1 & -17 & 21 & -1 &          100 \\
    0 &   1 &  2 & 11 & -\frac{1}{2} \\
    0 &   0 &  2 &  3 &            0 \\
    0 &   0 &  1 & -2 &            1
  \end{pmatrix}
\end{alignat*}

% ----------------------
\paragraph{Aufgabe 10.4} \textit{(Orthogonale Matrizen)}\par
% ----------------------
Eine Matrix $A\in M_{n\times n}(\mathbb{R})$ heißt orthogonal,
wenn $A^\text{t}\cdot A=I_n$ gilt.
\begin{enumerate}[a)]
  \item Zeigen Sie, dass für jede orthogonale Matrix $A$ die
        Gleichung $|\det(A)=1|$ gilt.
  \item Geben Sie mindestens vier verschiedene orthogonale Matrizen aus
        $M_{2\times2}(\mathbb{R})$ an.
\end{enumerate}

% ----------------------
\paragraph{Aufgabe 10.5} \textit{(Cramersche Regel)}\par
% ----------------------
Verwenden Sie die Cramersche Regel, um das lineare Gleichungssystem
$Ax=b$ über $\mathbb{R}$ zu lösen. Dabei seien $A$ und $b$ wie folgt gegeben:
\begin{equation*}
  A=
  \begin{pmatrix}
    1 &  2 & 3 \\
    2 & -1 & 1 \\
    3 &  2 & 4
  \end{pmatrix}
  \qquad
  b=
  \begin{pmatrix}
    1 \\
    0 \\
    1
  \end{pmatrix}
\end{equation*}
\begin{answer}
  \begin{equation*}
    \begin{split}
      x_1&=\frac{\det(A_1)}{\det(A)}\quad\text{mit}\quad
      A_1=
      \begin{pmatrix}
        1 &  2 & 3 \\
        0 & -1 & 1 \\
        1 &  2 & 4
      \end{pmatrix}
      \qquad
      x_2=\frac{\det(A_2)}{\det(A)}\quad\text{mit}\quad
      A_2=
      \begin{pmatrix}
        1 & 1 & 3 \\
        2 & 0 & 1 \\
        3 & 1 & 4
      \end{pmatrix}
      \\[2ex]
      x_3&=\frac{\det(A_3)}{\det(A)}\quad\text{mit}\quad
      A_3=
      \begin{pmatrix}
        1 &  2 & 1 \\
        2 & -1 & 0 \\
        3 &  2 & 1
      \end{pmatrix}
    \end{split}
  \end{equation*}

  \begin{equation*}
    \begin{split}
        \det(A)&=1\cdot(-1)\cdot4+2\cdot1\cdot3+3\cdot2\cdot2-3\cdot(-1)\cdot3-2\cdot1\cdot1-4\cdot2\cdot2=5\\
      \det(A_1)&=1\cdot(-1)\cdot4+2\cdot1\cdot1+3\cdot0\cdot2-1\cdot(-1)\cdot3-2\cdot1\cdot1-4\cdot0\cdot2=-1\\
      \det(A_2)&=1\cdot0\cdot4+1\cdot1\cdot3+3\cdot2\cdot1-3\cdot0\cdot3-1\cdot1\cdot1-4\cdot2\cdot1=0\\
      \det(A_3)&=1\cdot(-1)\cdot1+2\cdot0\cdot3+1\cdot2\cdot2-3\cdot(-1)\cdot1-2\cdot0\cdot1-1\cdot2\cdot2=2\\[2ex]
               &\Rightarrow\quad x=\frac{1}{5}\cdot
               \begin{pmatrix}
                 -1 \\
                  0 \\
                  2
               \end{pmatrix}
    \end{split}
  \end{equation*}
\end{answer}

% ----------------------
\paragraph{Aufgabe 11.1} \textit{(Eigenwerte)}\par
% ----------------------
Gegeben seien folgende linearen Abbildungen:
\begin{equation*}
  \begin{split}
    \ell&:\;\mathbb{C}^2\to\mathbb{C}^2\;,\;z\mapsto Az
    \quad\text{mit}\quad
    A=
    \begin{pmatrix}
      \imp{} & 0 \\
           2 & 1
    \end{pmatrix}
    \\[1ex]
    m&:\;\mathbb{Z}_5^4\to\mathbb{Z}_5^4\;,\;x\mapsto Bx
    \quad\text{mit}\quad
    B=
    \begin{pmatrix}
        7 & 5 & 3 & -3 \\
        5 & 1 & 2 &  2 \\
        4 & 0 & 1 &  2 \\
      -10 & 5 & 0 & 13
    \end{pmatrix}
  \end{split}
\end{equation*}
Bestimmen Sie die charakteristischen Polynome, Eigenwerte und Eigenvektoren
der beiden Abbildungen $\ell$ und $m$.
\begin{answer}
  \begin{equation*}
    \det(A-\lambda I)=\det
    \begin{pmatrix}
      \imp{}-\lambda &         0 \\
                   2 & 1-\lambda
    \end{pmatrix}
    \cdot
    \begin{pmatrix}
      x \\
      y
    \end{pmatrix}
    =(\imp{}-\lambda)(1-\lambda)-2\cdot0=0
    \quad\Rightarrow\quad
    \lambda_1=\imp{}\;,\;\lambda_2=1
  \end{equation*}

  \begin{equation*}
    \begin{split}
      \lambda_1=\imp{}\;&:\quad
      \begin{pmatrix}
        0 &        0 \\
        2 & 1-\imp{}
      \end{pmatrix}
      \cdot
      \begin{pmatrix}
        x \\
        y
      \end{pmatrix}
      =
      \begin{pmatrix}
        0 \\
        0
      \end{pmatrix}
      \quad\Rightarrow\quad
      v_1=t\cdot
      \begin{pmatrix}
        \imp{}-1 \\
               2
      \end{pmatrix}
      \quad\text{mit $t\in\mathbb{C}$}
      \\[2ex]
      \lambda_2=1\;&:\quad
      \begin{pmatrix}
        \imp{}-1 & 0 \\
               2 & 0
      \end{pmatrix}
      \cdot
      \begin{pmatrix}
        x \\
        y
      \end{pmatrix}
      =
      \begin{pmatrix}
        0 \\
        0
      \end{pmatrix}
      \quad\Rightarrow\quad
      v_2=t\cdot
      \begin{pmatrix}
        0 \\
        1
      \end{pmatrix}
      \quad\text{mit $t\in\mathbb{C}$}
    \end{split}
  \end{equation*}
\end{answer}

% ----------------------
\paragraph{Aufgabe 11.2} \textit{(Diagonalisierbar)}\par
% ----------------------
Gegeben seien folgende linearen Abbildungen:
\begin{alignat*}{4}
  f:&\;\; & \mathbb{R}^3&\to\mathbb{R}^3 & \quad&,\quad & (a,b,c)^\text{t}&\mapsto(2a+b ,\; b-c  ,\; 2b+4c)^\text{t} \\
  g:&\;\; & \mathbb{R}^3&\to\mathbb{R}^3 & \quad&,\quad & (a,b,c)^\text{t}&\mapsto(a+2b ,\; a+2b ,\; c)^\text{t}     \\
  h:&\;\; & \mathbb{R}^3&\to\mathbb{R}^3 & \quad&,\quad & (a,b,c)^\text{t}&\mapsto(a    ,\; -c   ,\; b)^\text{t}
\end{alignat*}
\begin{enumerate}[a)]
  \item Bestimmen Sie die charakteristischen Polynome, Eigenwerte und
        Eigenvektoren von $f$, $g$ und $h$.
  \item Entscheiden Sie jeweils, ob $f$, $g$ und $h$ diagonalisierbar sind.
\end{enumerate}
\begin{answer}
  Abbildung $f$:
  \begin{equation*}
    f
    \begin{pmatrix}
      a \\
      b \\
      c
    \end{pmatrix}
    =
    \begin{pmatrix}
      2 & 1 &  0 \\
      0 & 1 & -1 \\
      0 & 2 &  4
    \end{pmatrix}
    \cdot
    \begin{pmatrix}
      a \\
      b \\
      c
    \end{pmatrix}
  \end{equation*}

  Charakteristisches Polynom und Eigenwerte:
  \begin{equation*}
    \begin{split}
      \det
      \begin{pmatrix}
        2-\lambda & 1 &  0 \\
        0 & 1-\lambda & -1 \\
        0 & 2 &  4-\lambda
      \end{pmatrix}
      &=(2-\lambda)(1-\lambda)(4-\lambda)-2\cdot(-1)\cdot(2-\lambda)=0\\
      &=(2-\lambda)\Big[(1-\lambda)(4-\lambda)+2\Big]\\
      &=(2-\lambda)\Big[\lambda^2-5\lambda+6\Big]\\
      &=(2-\lambda)\left[\lambda^2-5\lambda+\left(\frac{5}{2}\right)^2-\left(\frac{5}{2}\right)^2+6\right]\\
      &=(2-\lambda)\left[\left(\lambda-\frac{5}{2}\right)^2-\frac{1}{4}\right]\\
      &=(2-\lambda)(\lambda-2)(\lambda-3)\\[1ex]
      &\Rightarrow \lambda_1=2\;,\;\lambda_2=2\;,\;\lambda_3=3
    \end{split}
  \end{equation*}

  Eigenvektoren:
  \begin{equation*}
    \begin{pmatrix}
      2-\lambda & 1 &  0 \\
      0 & 1-\lambda & -1 \\
      0 & 2 &  4-\lambda
    \end{pmatrix}
    \cdot
    \begin{pmatrix}
      x \\
      y \\
      z
    \end{pmatrix}
    =
    \begin{pmatrix}
      0 \\
      0 \\
      0
    \end{pmatrix}
  \end{equation*}

  \begin{equation*}
    \begin{split}
      \lambda_1=\lambda_2=2\;&:\quad
      \begin{pmatrix}
        0 &  1 &  0 \\
        0 & -1 & -1 \\
        0 &  2 &  2
      \end{pmatrix}
      \cdot
      \begin{pmatrix}
        x \\
        y \\
        z
      \end{pmatrix}
      =
      \begin{pmatrix}
        0 \\
        0 \\
        0
      \end{pmatrix}
      \quad\Rightarrow\quad
      v_1=v_2=t\cdot
      \begin{pmatrix}
        1 \\
        0 \\
        0
      \end{pmatrix}
      \quad\text{mit $t\in\mathbb{R}$}
      \\[2ex]
      \lambda_3=3\;&:\quad
      \begin{pmatrix}
        -1 &  1 &  0 \\
         0 & -2 & -1 \\
         0 &  2 &  1
      \end{pmatrix}
      \cdot
      \begin{pmatrix}
        x \\
        y \\
        z
      \end{pmatrix}
      =
      \begin{pmatrix}
        0 \\
        0 \\
        0
      \end{pmatrix}
      \quad\Rightarrow\quad
      v_3=t\cdot
      \begin{pmatrix}
         1 \\
         1 \\
        -2
      \end{pmatrix}
      \quad\text{mit $t\in\mathbb{R}$}
    \end{split}
  \end{equation*}
  Diagonalisierbar: Nein, zwei Vektoren können keine Basis des $\mathbb{R}^3$ bilden.

  Abbildung $g$:
  \begin{equation*}
    g
    \begin{pmatrix}
      a \\
      b \\
      c
    \end{pmatrix}
    =
    \begin{pmatrix}
      1 & 2 & 0 \\
      1 & 2 & 0 \\
      0 & 0 & 1
    \end{pmatrix}
    \cdot
    \begin{pmatrix}
      a \\
      b \\
      c
    \end{pmatrix}
  \end{equation*}

  Charakteristisches Polynom und Eigenwerte:
  \begin{equation*}
    \det
    \begin{pmatrix}
      1-\lambda & 2 & 0 \\
      1 & 2-\lambda & 0 \\
      0 & 0 & 1-\lambda
    \end{pmatrix}
    =(1-\lambda)\cdot\lambda\cdot(\lambda-3)
  \end{equation*}

  Eigenvektoren:
  \begin{equation*}
    \lambda_1=1:
    t\cdot
    \begin{pmatrix}
      0 \\
      0 \\
      1
    \end{pmatrix}
    \qquad
    \lambda_2=0:
    t\cdot
    \begin{pmatrix}
      -2 \\
       1 \\
       0
    \end{pmatrix}
    \qquad
    \lambda_3=3:
    t\cdot
    \begin{pmatrix}
      1 \\
      1 \\
      0
    \end{pmatrix}
    \qquad
    \text{mit $t\in\mathbb{R}$}
  \end{equation*}

  Diagonalisierbar: Ja, weil die Eigenvektoren eine Basis des $\mathbb{R}^3$ bilden!

  Abbildung $h$:
  \begin{equation*}
    h
    \begin{pmatrix}
      a \\
      b \\
      c
    \end{pmatrix}
    =
    \begin{pmatrix}
      1 & 0 &  0 \\
      0 & 0 & -1 \\
      0 & 1 &  0
    \end{pmatrix}
    \cdot
    \begin{pmatrix}
      a \\
      b \\
      c
    \end{pmatrix}
  \end{equation*}

  Charakteristisches Polynom und Eigenwerte:
  \begin{equation*}
    \det
    \begin{pmatrix}
      1-\lambda & 0 &  0 \\
      0 & 0-\lambda & -1 \\
      0 & 1 &  0-\lambda
    \end{pmatrix}
    =(1-\lambda)(\lambda^2+1)
    \qquad
    \lambda_1=1,\;\lambda_2=\imp{},\;\lambda_3=-\imp{}
  \end{equation*}

  Eigenvektoren:
  \begin{equation*}
    \lambda_1=1:
    t\cdot
    \begin{pmatrix}
      1 \\
      0 \\
      0
    \end{pmatrix}
    \qquad
    \lambda_2=\imp{}:
    t\cdot
    \begin{pmatrix}
           0 \\
      \imp{} \\
           1
    \end{pmatrix}
    \qquad
    \lambda_3=-\imp{}:
    t\cdot
    \begin{pmatrix}
           0 \\
           1 \\
      \imp{}
    \end{pmatrix}
    \qquad
    \text{mit $t\in\mathbb{C}$}
  \end{equation*}

  Diagonalisierbar: Nein, weil in den Eigenvektoren imaginäre Zahlen stehen, die
  auf keinen Fall eine Basis des $\mathbb{R}^3$ bilden können.
\end{answer}

% ----------------------
\paragraph{Aufgabe 11.3} \textit{(Matrix zu gegebenen Eigenwerten bestimmen)}\par
% ----------------------
Bestimmen Sie eine Matrix $A\in M_{5\times5}(\mathbb{R})$ welche folgende
Eigenwerte besitzt:
\begin{equation*}
  \lambda_1=1
  \qquad
  \lambda_2=\lambda_3=2
  \qquad
  \lambda_4=20
  \qquad
  \lambda_5=-\frac{1}{2}
\end{equation*}
\begin{answer}
  \begin{equation*}
    \text{z.\,B.}\quad
    M=
    \begin{pmatrix}
      1 & 0 & 0 &  0 &            0 \\
      0 & 2 & 0 &  0 &            0 \\
      0 & 0 & 2 &  0 &            0 \\
      0 & 0 & 0 & 20 &            0 \\
      0 & 0 & 0 &  0 & -\frac{1}{2}
    \end{pmatrix}
  \end{equation*}
\end{answer}

% ----------------------
\paragraph{Aufgabe 11.4} \textit{(Matrix gesucht)}\par
% ----------------------
Bestimmen Sie eine Matrix $A\in M_{2\times2}(\mathbb{R})$ über die
folgendes bekannt ist:
\begin{equation*}
  \begin{split}
    \begin{pmatrix}
      \frac{3}{2} \\
                1
    \end{pmatrix}
    &\text{ ist Eigenvektor zum Eigenwert }
    \lambda_1=3
    \\[1ex]
    \begin{pmatrix}
       1 \\
      -1
    \end{pmatrix}
    &\text{ ist Eigenvektor zum Eigenwert }
    \lambda_2=-2
  \end{split}
\end{equation*}
\begin{answer}
  \begin{equation*}
    M=
    \begin{pmatrix}
      a & b \\
      c & d
    \end{pmatrix}
    \quad
    \Rightarrow
    \quad
    \begin{cases}
      \begin{pmatrix}
        a & b \\
        c & d
      \end{pmatrix}
      \cdot
      \begin{pmatrix}
         \frac{3}{2} \\
                   1
      \end{pmatrix}
      =
      3\cdot
      \begin{pmatrix}
         \frac{3}{2} \\
                   1
      \end{pmatrix}
      \\[2ex]
      \begin{pmatrix}
        a & b \\
        c & d
      \end{pmatrix}
      \cdot
      \begin{pmatrix}
         1 \\
        -1
      \end{pmatrix}
      =
      -2\cdot
      \begin{pmatrix}
         1 \\
        -1
      \end{pmatrix}
    \end{cases}
    \quad
    \Rightarrow
    \quad
    M=
    \begin{pmatrix}
      1 & 3 \\
      2 & 0
    \end{pmatrix}
  \end{equation*}
\end{answer}

% ----------------------
\paragraph{Aufgabe 12.1} \textit{(Skalarprodukt)}\par
% ----------------------
Gegeben sei eine symmetrische Matrix $S\in M_{n\times n}(\mathbb{R})$.
Zeigen Sie, dass durch
\begin{equation*}
  \langle x,y\rangle\defeq x^\text{t}\cdot S\cdot y
\end{equation*}
ein Skalarprodukt definiert wird, wenn
$\forall x\in\mathbb{R}^n\setminus\{0\}:\langle x,x\rangle>0$.
\begin{answer}
  Bilinearität:
  \begin{equation*}
    \begin{split}
      \langle u+v,w\rangle
      &=(u+v)^\text{t}\cdot S\cdot w\\
      &=u^\text{t}\cdot S\cdot w + v^\text{t}\cdot S\cdot w\\
      &=\langle u,w\rangle+\langle v,w\rangle
    \end{split}
    \qquad
    \begin{split}
      \langle\lambda\cdot u,v\rangle
      &=(\lambda\cdot u)^\text{t}\cdot S\cdot v\\
      &=\lambda\cdot u^\text{t}\cdot S\cdot v\\
      &=\lambda\cdot\langle u,v\rangle
    \end{split}
  \end{equation*}
  \begin{equation*}
    \begin{split}
      \langle u,v+w\rangle
      &=u^\text{t}\cdot S\cdot(v+w)\\
      &=u^\text{t}\cdot S\cdot v + u^\text{t}\cdot S\cdot w\\
      &=\langle u,v\rangle+\langle u,w\rangle
    \end{split}
    \qquad
    \begin{split}
      \langle u,\lambda\cdot v\rangle
      &=u^\text{t}\cdot S\cdot(\lambda\cdot v)\\
      &=\lambda\cdot u^\text{t}\cdot S\cdot v\\
      &=\lambda\cdot\langle u,v\rangle
    \end{split}
  \end{equation*}
  \par
  Symmetrie:
  \begin{equation*}
    \begin{split}
      \langle u,w\rangle
      &=u^\text{t}\cdot S\cdot w\\
      &=\Big((u^\text{t}\cdot S\cdot w)^\text{t}\Big)^\text{t}\\
      &=\Big((S\cdot w)^\text{t}\cdot u\Big)^\text{t}\\
      &=\Big(w^\text{t}\cdot S^\text{t}\cdot u\Big)^\text{t}\\
      &=\Big(w^\text{t}\cdot S\cdot u\Big)^\text{t}\qquad\text{hier gilt: $(w^\text{t}\cdot S\cdot u)\in\mathbb{R}$}\\
      &=w^\text{t}\cdot S\cdot u\\
      &=\langle w,u\rangle
    \end{split}
  \end{equation*}
  \par
  Positivdefinitheit: Ist laut Voraussetzungen schon gegeben.
\end{answer}

% -------------------------------------------------------------------
\paragraph{Probeklausur 30.07.2018 -- Aufgabe 1} \textit{[10 Punkte]}\par
% -------------------------------------------------------------------
Schreiben Sie die folgenden komplexen Zahlen in der Form $a+\imp{b}$,
mit $a,b\in\mathbb{R}$ und bestimmen Sie das multiplikative Inverse
auch in dieser Form.
\begin{enumerate}[a)]
  \item $\displaystyle-\frac{1}{\imp{}}$
  \item $\displaystyle\frac{2}{\imp{}}+\frac{\imp{}}{1}$
  \item Finden Sie außerdem alle $x$, die die folgende Gleichung lösen
        \begin{equation*}
          \frac{1+x}{1-x}=\imp{}
        \end{equation*}
\end{enumerate}

% ------------------------------------------------------------------
\paragraph{Probeklausur 30.07.2018 -- Aufgabe 2} \textit{[8 Punkte]}\par
% ------------------------------------------------------------------
Seien $f,g\in\mathbb{Q}[X]$ gegeben durch
\begin{equation*}
  \begin{split}
    f(X)&=5X^4+6X^2+3X-10\quad\text{und}\\
    g(X)&=5X+2.
  \end{split}
\end{equation*}
Finden Sie mit Hilfe des Verfahrens des Beweises von Satz 6
Polynome $q$ und $r$, sodass
\begin{equation*}
  f=qg+r\quad\text{und}\quad\deg(r)<\deg(g).
\end{equation*}

% ------------------------------------------------------------------
\paragraph{Probeklausur 30.07.2018 -- Aufgabe 3} \textit{[8 Punkte]}\par
% ------------------------------------------------------------------
Lösen Sie das folgende lineare Gleichungssystem über $\mathbb{R}$
mit dem Verfahren von Gauß:
\begin{equation*}
  \left(
  \begin{array}{rrrr|r}
    1 & 2 & 3 & 4 & 6 \\
    0 & 1 & 2 & 2 & 3 \\
    2 & 1 & 0 & 1 & 2 \\
    3 & 2 & 1 & 3 & 5 \\
    4 & 3 & 2 & 1 & 4
  \end{array}
  \right)
\end{equation*}

% -------------------------------------------------------------------
\paragraph{Probeklausur 30.07.2018 -- Aufgabe 4} \textit{[10 Punkte]}\par
% -------------------------------------------------------------------
Beweisen oder widerlegen Sie:
Für Matrizen $A,B\in M_{n\times n}(\mathbb{R})$ mit $n\geq2$ gilt:
\begin{equation*}
  (A+B)^2=A^2+2AB+B^2
\end{equation*}

% -------------------------------------------------------------------
\paragraph{Probeklausur 30.07.2018 -- Aufgabe 5} \textit{[10 Punkte]}
% -------------------------------------------------------------------
\begin{enumerate}[a)]
  \item Seien $v_1,v_2,v_3\in V$ linear unabhängige Vektoren in einem
        $\mathbb{R}$-Vektorraum. Zeigen Sie, dass dann auch
        \begin{equation*}
          w_1=v_1-v_2
          \qquad
          w_2=v_2-v_3
          \qquad\text{und}\qquad
          w_3=v_3+v_1
        \end{equation*}
        linear unabhängig sind.
  \item Seien $v_1,v_2,v_3\in V$ beliebige Vektoren in einem
        $\mathbb{R}$-Vektorraum. Zeigen Sie, dass
        \begin{equation*}
          w_1=v_1-v_2
          \qquad
          w_2=v_2-v_3
          \qquad\text{und}\qquad
          w_3=v_1-v_3
        \end{equation*}
        immer linear abhängig sind.
\end{enumerate}

% -------------------------------------------------------------------
\paragraph{Probeklausur 30.07.2018 -- Aufgabe 6} \textit{[10 Punkte]}\par
% -------------------------------------------------------------------
Seien $U_1$ und $U_2$ Unterräume eines $K$-Vektorraumes $V$.
Beweisen oder widerlegen Sie:
\begin{enumerate}[a)]
  \item $U_1\cap U_2$ ist ebenfalls Unterraum.
  \item $U_1\cup U_2$ ist ebenfalls Unterraum.
  \item $U_1\setminus U_2$ ist ebenfalls Unterraum.
\end{enumerate}

% -------------------------------------------------------------------
\paragraph{Probeklausur 30.07.2018 -- Aufgabe 7} \textit{[10 Punkte]}\par
% -------------------------------------------------------------------
Sei $V$ der Vektorraum aller Funktionen $\mathbb{R}\to\mathbb{R}$.
Welche der folgenden Teilmengen sind Unterräume? (Ohne Beweis)
\begin{enumerate}[a)]
  \item $M_1=\{f\in V\mid f(1)=f(-1)\}$
  \item $M_2=\{f\in V\mid f(0)\cdot f(1)=0\}$
  \item $M_3=\{f\in V\mid f(0)=r10\}$
\end{enumerate}

% -------------------------------------------------------------------
\paragraph{Probeklausur 31.07.2018 -- Aufgabe 1} \textit{[10 Punkte]}\par
% -------------------------------------------------------------------
Bestimmen Sie den Rang der folgenden Matrix.
\begin{equation*}
  \begin{pmatrix}
     1 &  4 &  5 & 0 \\
    -1 &  0 & -1 & 0 \\
     3 & -2 &  1 & 0 \\
     2 &  3 &  5 & 1
  \end{pmatrix}
\end{equation*}
\begin{answer}
  \newcommand{\matnum}[1]{\makebox[0pt][r]{{\small#1}\hspace*{1.5em}}}%
  \newcommand{\matmod}[1]{\makebox[0pt][l]{\hspace*{1.5em}\ensuremath{|{}#1}}}%
  \newcolumntype{N}{p{0pt}}%
  \newcolumntype{B}{>{\hspace*{\fill}}p{1em}}%
  \newcolumntype{R}{>{\hspace*{\fill}}p{2em}}%
  \newcolumntype{E}{p{2pt}}%
  \setlength{\arraycolsep}{0pt}%
  \begin{equation*}
    \begin{split}
      &\left(
      \begin{array}{NBRRRE}
        \matnum{I}   &  1 &  4 &  5 & 0 &                          \\
        \matnum{II}  & -1 &  0 & -1 & 0 & \matmod{+\text{I}}       \\
        \matnum{III} &  3 & -2 &  1 & 0 & \matmod{-3\cdot\text{I}} \\
        \matnum{IV}  &  2 &  3 &  5 & 1 & \matmod{-2\cdot\text{I}}
      \end{array}
      \right)\\[2ex]
      &\left(
      \begin{array}{NBRRRE}
        \matnum{I}   & 1 &   4 &   5 & 0 & \\
        \matnum{II}  & 0 &   4 &   4 & 0 & \\
        \matnum{III} & 0 & -14 & -14 & 0 & \\
        \matnum{IV}  & 0 &  -5 &  -5 & 1 &
      \end{array}
      \right)
    \end{split}
  \end{equation*}
  \begin{mytemize}
    \item An dieser Stelle erkennt man schon, dass die Zeilen II und III
          (als einzige) linear abhängig sind. Der Rang der Matrix beträgt also 3.
  \end{mytemize}
\end{answer}

% ------------------------------------------------------------------
\paragraph{Probeklausur 31.07.2018 -- Aufgabe 2} \textit{[6 Punkte]}\par
% ------------------------------------------------------------------
Bestimmen Sie mit Hilfe des Dimensionssatzes $\dim(\ker(\varphi))$ für
\begin{enumerate}[a)]
  \item $\varphi:K^6\to K^3$, mit $\varphi$ surjektiv
  \item $\varphi:K^4\to K^7$, mit $\dim(\operatorname{bild}(\varphi))=3$
  \item $\varphi:M_{2\times2}(K)\to M_{2\times2}(K)$, mit $\dim(\operatorname{bild}(\varphi))=1$
\end{enumerate}

% ------------------------------------------------------------------
\paragraph{Probeklausur 31.07.2018 -- Aufgabe 3} \textit{[8 Punkte]}\par
% ------------------------------------------------------------------
Bestimmen Sie mit Hilfe der Determinanten, für welche $k\in\mathbb{R}$
die Matrix
\begin{equation*}
  \begin{pmatrix}
    k & 1 & 0 & -k \\
    0 & 1 & 0 &  0 \\
    0 & 1 & 1 &  0 \\
    k & 0 & 0 &  1
  \end{pmatrix}
  \in M_{4\times4}(\mathbb{R})
\end{equation*}
invertierbar ist.
\begin{answer}
  \begin{mytemize}
    \item Die Entwicklung der Determinanten nach einer Zeile erfolge gemäß der Gleichung
          \begin{equation*}
            \det(A)=\sum_{j=1}^n(-1)^{i+j}a_{ij}\det(A_{ij})\;,
          \end{equation*}
          wobei $A_{ij}$ die Matrix bezeichnet, die aus $A$ entsteht, wenn man die
          $i$-te Zeile und die $j$-te Spalte streicht.
    \item Die Entwicklung nach der 2. Zeile ergibt hier also:
          \begin{equation*}
            \begin{split}
              \det(A)&=\sum_{j=1}^4(-1)^{2+j}a_{2j}\det(A_{2j})
                      =(-1)^{2+2}a_{22}\det(A_{22})\\[1ex]
              &=\det
              \begin{pmatrix}
                k & 0 & -k \\
                0 & 1 &  0 \\
                k & 0 &  1
              \end{pmatrix}
              =k+k^2=k(1+k)
            \end{split}
          \end{equation*}
    \item Eine Matrix ist genau dann invertierbar, falls die Determinante nicht Null ist:
          \begin{equation*}
              0\neq k(1+k)\quad\Rightarrow\quad k\in\mathbb{R}\setminus\{0,-1\}
          \end{equation*}
  \end{mytemize}
\end{answer}

% ------------------------------------------------------------------
\paragraph{Probeklausur 31.07.2018 -- Aufgabe 4} \textit{[8 Punkte]}\par
% ------------------------------------------------------------------
Finden Sie das Inverse zu folgender Matrix:
\begin{equation*}
  \begin{pmatrix}
    1 & 1 & 0 \\
    1 & 1 & 1 \\
    0 & 1 & 1
  \end{pmatrix}
\end{equation*}
\begin{answer}
  \newcommand{\matnum}[1]{\makebox[0pt][r]{{\small#1}\hspace*{1.5em}}}%
  \newcommand{\matmod}[1]{\makebox[0pt][l]{\hspace*{1.5em}\ensuremath{|{}#1}}}%
  \newcolumntype{N}{p{0pt}}%
  \newcolumntype{B}{>{\hspace*{\fill}}p{0.6em}}%
  \newcolumntype{R}{>{\hspace*{\fill}}p{1.5em}}%
  \newcolumntype{E}{p{2pt}}%
  \newcolumntype{S}{p{10pt}}%
  \setlength{\arraycolsep}{0pt}%
  \begin{equation*}
    \begin{split}
      &\left(
      \begin{array}{NBRRS|RRRE}
        \matnum{I}   & 1 & 1 & 0 && 1 & 0 & 0 &                    \\
        \matnum{II}  & 1 & 1 & 1 && 0 & 1 & 0 & \matmod{-\text{I}} \\
        \matnum{III} & 0 & 1 & 1 && 0 & 0 & 1 &
      \end{array}
      \right)\\[2ex]
      &\left(
      \begin{array}{NBRRS|RRRE}
        \matnum{I}   & 1 & 1 & 0 &&  1 & 0 & 0 &                               \\
        \matnum{II}  & 0 & 0 & 1 && -1 & 1 & 0 & \matmod{\leftarrow\text{III}} \\
        \matnum{III} & 0 & 1 & 1 &&  0 & 0 & 1 & \matmod{\leftarrow\text{II}}
      \end{array}
      \right)\\[2ex]
      &\left(
      \begin{array}{NBRRS|RRRE}
        \matnum{I}   & 1 & 1 & 0 &&  1 & 0 & 0 &                      \\
        \matnum{II}  & 0 & 1 & 1 &&  0 & 0 & 1 & \matmod{-\text{III}} \\
        \matnum{III} & 0 & 0 & 1 && -1 & 1 & 0 &
      \end{array}
      \right)\\[2ex]
      &\left(
      \begin{array}{NBRRS|RRRE}
        \matnum{I}   & 1 & 1 & 0 &&  1 &  0 & 0 & \matmod{-\text{II}} \\
        \matnum{II}  & 0 & 1 & 0 &&  1 & -1 & 1 &                     \\
        \matnum{III} & 0 & 0 & 1 && -1 &  1 & 0 &
      \end{array}
      \right)\\[2ex]
      &\left(
      \begin{array}{NBRRS|RRRE}
        \matnum{I}   & 1 & 0 & 0 &&  0 &  1 & -1 & \\
        \matnum{II}  & 0 & 1 & 0 &&  1 & -1 &  1 & \\
        \matnum{III} & 0 & 0 & 1 && -1 &  1 &  0 &
      \end{array}
      \right)\\[2ex]
    \end{split}
  \end{equation*}
\end{answer}

% ------------------------------------------------------------------
\paragraph{Probeklausur 31.07.2018 -- Aufgabe 5} \textit{[6 Punkte]}\par
% ------------------------------------------------------------------
Seien $A$, $B$ und $S$ $n\times n$ Matrizen, $S$ invertierbar und $A=S^{-1}BS$.
Begründen Sie ausführlich, warum $\det(A)=\det(B)$ gilt.
\begin{answer}
  \begin{equation*}
    \begin{split}
      \det(A)&=\det(S^{-1}BS)\\
      &=\det(S^{-1})\cdot\det(B)\cdot\det(S)\qquad\text{dies sind drei reelle Zahlen}\\
      &=\det(S)^{-1}\cdot\det(S)\cdot\det(B)\\[1ex]
      &=\frac{\det(S)}{\det(S)}\cdot\det(B)=\det(B)
    \end{split}
  \end{equation*}
\end{answer}

% ------------------------------------------------------------------
\paragraph{Probeklausur 31.07.2018 -- Aufgabe 6} \textit{[9 Punkte]}\par
% ------------------------------------------------------------------
Zeigen Sie:
\begin{enumerate}[a)]
  \item Ist $A\in M_{n\times n}(K)$ eine invertierbare Matrix und $k\in K$
        mit $k\neq0$, so ist auch $kA$ invertierbar.
  \item Ist $A^2=0$, so sind auch $I_n-A$ und $I_n+A$ invertierbar.
\end{enumerate}
Geben Sie außerdem ein Beispiel an, das zeigt, dass die folgende
Aussage falsch ist:
\begin{enumerate}[a)]
  \setcounter{enumi}{2}
  \item Sind $A,B\in M_{n\times n}(K)$ invertierbar, so ist auch $A+B$
        invertierbar.
\end{enumerate}
\begin{answer}
  Begründung zu a)
  \begin{equation*}
    \det(k\cdot A)=k^n\cdot\det(A)
  \end{equation*}
  \begin{mytemize}
    \item weil $A$ nach Vorraussetzung invertierbar ist, gilt $\det(A)\neq0$
    \item nach Vorraussetzung ist $k\neq0$, also auch $k^n\neq0$
    \item also ist auch $\det(k\cdot A)\neq0$ und $k\cdot A$ damit invertierbar
  \end{mytemize}
  \par
  Begründung zu b)
  \begin{equation*}
    \begin{split}
      A^2=0&\Rightarrow\det\big(I_n-A^2\big)\neq0\\
           &\Rightarrow\det\big(I_n-A+A-A^2\big)\neq0\\
           &\Rightarrow\det\big((I_n-A)\cdot I_n+(I_n-A)\cdot A\big)\neq0\\
           &\Rightarrow\det\big((I_n-A)\cdot(I_n+A)\big)\neq0\\
           &\Rightarrow\det(I_n-A)\cdot\det(I_n+A)\neq0\\
           &\Rightarrow(I_n\pm A)\text{ sind invertierbar}
    \end{split}
  \end{equation*}
  \begin{mytemize}
    \item das Ausmultiplizieren von $(I_n-A)\cdot(I_n+A)$ hat zur Lösung geführt\ldots
  \end{mytemize}
  \par
  Gegenbeispiel zu c)
  \begin{equation*}
    A=
    \begin{pmatrix}
      1 & 0 & 0 \\
      0 & 1 & 0 \\
      0 & 0 & 1
    \end{pmatrix}
    \quad
    \det(A)=1
    \qquad
    \qquad
    B=
    \begin{pmatrix}
      1 &  0 & 0 \\
      0 & -1 & 0 \\
      0 &  0 & 1
    \end{pmatrix}
    \quad
    \det(B)=-1
  \end{equation*}
  \begin{mytemize}
    \item beide Matritzen sind invertierbar, da beide Determinanten ungleich Null sind
  \end{mytemize}
  \begin{equation*}
    \det(A+B)=\det
    \begin{pmatrix}
      2 & 0 & 0 \\
      0 & 0 & 0 \\
      0 & 0 & 2
    \end{pmatrix}
    =0
  \end{equation*}
  \begin{mytemize}
    \item also ist die Matrix $C=A+B$ nicht invertierbar
  \end{mytemize}
\end{answer}

% -------------------------------------------------------------------
\paragraph{Probeklausur 31.07.2018 -- Aufgabe 7} \textit{[12 Punkte]}\par
% -------------------------------------------------------------------
Sei $\varphi:V\to W$ eine lineare Abbildung zwischen Vektorräumen.
Beweisen oder widerlegen Sie:
\begin{enumerate}[a)]
  \item Sind $v_1,\ldots,v_n$ Eigenvektoren mit dem gleichen Eigenwert $k$,
        so ist jeder nicht-triviale Vektor aus $\langle v_1,\ldots,v_n\rangle$
        Eigenvektor zum Eigenwert $k$.
  \item Ist $v$ ein Eigenvektor mit Eigenwert $k$ und $w$ ein Eigenvektor
        mit Eigenwert $\ell$, so ist $v+w$ Eigenvektor mit Eigenwert $k+\ell$.
  \item Ist $v$ ein Eigenvektor von $\varphi$ und gleichzeitig auch Eigenvektor
        von $\psi:W\to U$, so ist $v$ auch Eigenvektor von $\psi\circ\varphi$.
  \item Ist $k$ ein Eigenwert von $\varphi$ und gleichzeitig auch Eigenwert von
        $\psi:W\to U$, so ist $k$ auch Eigenwert von $\psi\circ\varphi$.
\end{enumerate}
\begin{answer}
  Aussage b) ist falsch:
  \begin{equation*}
    A=
    \begin{pmatrix}
      k &    0 \\
      0 & \ell
    \end{pmatrix}
    \quad
    \text{diese Matrix besitzt gar keinen Eigenwert $k+\ell$}
  \end{equation*}
  Aussage c) ist wahr:
  \begin{equation*}
    \begin{split}
      \varphi(v)&=\lambda\cdot v\\
      \psi(v)&=\mu\cdot v\\
      \psi(\varphi(v))&=\psi(\lambda v)=\lambda\cdot\psi(v)=\lambda\mu\cdot v
    \end{split}
  \end{equation*}
  Aussage d) ist falsch:
  \begin{equation*}
    \begin{split}
      \varphi(v)&=k\cdot v\\
      \psi(w)&=k\cdot w\\
      \psi(\varphi(v))&=\psi(k v)=k\cdot\psi(v)=k^2\cdot v
    \end{split}
  \end{equation*}
\end{answer}

% -------------------------------------------------------------------
\paragraph{Probeklausur 31.07.2018 -- Aufgabe 8} \textit{[16 Punkte]}\par
% -------------------------------------------------------------------
Finden Sie alle Eigenwerte der folgenden Matrix $A\in M_{3\times3}(\mathbb{R})$
und bestimmen Sie zu jedem Eigenwert einen Eigenvektor.
\begin{equation*}
  A=
  \begin{pmatrix}
     1 & -1 &  0 \\
    -1 &  2 & -1 \\
     0 & -1 &  1
  \end{pmatrix}
\end{equation*}

% ------------------------------------------------------------------------------
\end{document}
% ------------------------------------------------------------------------------
